At the meeting of the Initiative for Cuneiform Encoding in 2000, it
was determined that “the encoding of proto-cuneiform is initially, but
not neces⟨s⟩arily ultimately, beyond the purview of the project
cuneiform encoding” \cite{L2/00-398}. When the principles for encoding
the cuneiform script were set out in \cite{L2/03-162}, the question of
unification with the script of the archaic period was left open.

The initial scope of the encoding covered the cuneiform script as used
from the Ur III period through the first millennium, although a few
distinctions only apparent in the Early Dynastic period, such as 𒅎 IM
vs. 𒉎 NI₂, and a few characters used only in the Early Dynastic
period, such as 𒍡 ZAMₓ(ELLes 396), were also captured in the initial
encoding in Unicode Version 5.0. The scope of the cuneiform script was
expanded two and a half centuries back in time in Unicode Version 8.0
with the addition of the Early Dynastic Cuneiform block, based on the
Early Dynastic IIIa repertoire in \cite{LAK}; see the
proposal \cite{L2/12-208}.

At first glance, it may therefore seem possible to expand the scope of
the cuneiform script further to the fourth millennium. The expansion
of the glyphic range would be minor compared to its current extent;
consider the variety of glyphs for U+1214B {\oraccnoto 𒅋} CUNEIFORM
SIGN IL, from Early Dynastic IIIb {\oraccrsp 𒅋} to Old Babylonian
{\oraccobf 𒅋} and Neo-Assyrian {\oraccnao 𒅋}. Indeed the glyphs of
some proto-cuneiform signs would lie within the glyphic range of some
archaizing Early Dynastic inscriptions: compare the sign LAK500
{\oracclak 𒅋} or {\oracclaksaltiii 𒅋} on the shoulder of the ED IIIb
statue of {\oraccnoto 𒂗𒋾𒅋} \cite{P225850} with the reference glyph
proposed for U+12860 {\pcxxv 𒡠} PROTO-CUNEIFORM SIGN IL.

% NOTE(egg): One would expect Oracc LAK to make for another good
% example here, but it does not have the ED IIIa LAK500, and instead
% uses an OAkk glyph (LAK501=REC314) for 𒅋.
%
% NOTE(sjt): LAK500 is now the glyph for u1214B in Oracc-LAK;
% Ebih-il's IL is u1214B.3 in Oracc-LAK.

This impression of unifiability is reinforced by the treatment of the
Uruk and Jemdet Nasr forms as another column in the
diachronic \cite{MÉA}, and the concordances with \cite{LAK} and
multiple Sumerian readings for each sign given
in \cite[167--346]{ATU2}, as in any other cuneiform sign list.

However, a discontinuity in the approach to sign identity in
proto-cuneiform appears in \cite[347\psqq]{ATU2}. Wary of letting a
Sumerian reading based on third millennium texts hide distinctions
specific to the fourth millennium corpus, Englund makes major changes
to the identification of signs, which have been followed in
proto-cuneiform studies since:

1. Switching to a system of opaque sign names without interpretation,
for instance, always GAR for the proposed U+127EF 𒟯 PROTO-CUNEIFORM
SIGN GAR regardless of context rather than a context-dependent reading
gar (heap), niŋ₂ (thing), or ninda (bread) and interchangeable GAR,
NIŊ₂, or NINDA when referring to the abstract sign for U+120FB 𒃻
CUNEIFORM SIGN GAR.

2. Avoiding sign names based on Sumerian readings of sign sequences,
so that 𒤡𒨟 is generally transliterated as a sequence, MUŠEN ŠEa,
rather than UZa.

3. Classifying allographs, for instance, KAŠa through KAŠd for
variants of signs representing pots, whose cuneiform reflex is likely
U+12049 𒁉 CUNEIFORM SIGN BI (=KAŠ).

The classification of allographs can reveal semantic contrasts; for
instance, the signs 𒢄 KAŠa and 𒢅 KAŠb appear to correspond to pots
containing different substances, see \cite[168]{Englund1998}, even though
the linguistic distinction is unknown. In other cases, no semantic
contrast can be identified, as between 𒢅 KAŠb and 𒢇 KAŠc. Effectively,
from a character encoding standpoint, this approach to character
identity is more akin to the one used for undeciphered scripts, and is
contradictory to the model for a fully-deciphered script, where
orthographic distinctions are encodable, but stylistic ones are not,
even when they are somewhat systematic.

The avoidance of readings is also incompatible with a unified model,
because it conflicts with the diachronic handling of mergers and
splits. For signs that split in later phases of cuneiform, but have
identical appearances in the early third millennium, such as 𒈩 MES and
𒁾 DUB, an interoperable encoding is obtained based on the reading: a
sign read mes is encoded as U+12229, and a sign read dub is encoded as
U+1207E. However, if the proto-cuneiform approach is followed and the
sign is invariably transliterated as DUB, the resulting encoding may
be incompatible with the one based on a Sumerian reading; see the
example of the personal name 𒈩𒇽𒉡𒂠 mes-lu₂-nu-še₃ in Section \ref{ED
I-II Excluded from the Proposal}.

The same holds for the avoidance of readings of sign sequences; an
interoperable encoding of cuneiform 𒉭𒊻𒄷 is achieved by reading it as
nunuz uzmušen “duck eggs”, and accordingly using the UZ sign, U+122BB
𒊻, whereas 𝑛𒋡𒊺𒄷𒊺 is read as 𝑛 sila₃ še mušen niga “𝑛 sila of barley
for fattening the birds”, and thus encoded using the sequence ⟨U+122BA
𒊺, U+12137 𒄷⟩. In general, this means that in the cuneiform script,
interoperability is retained regardless of whether a particular diri
is encoded as a sequence, like 𒋛𒀀 or 𒄷𒋛, or atomically, like 𒊻 or
𒎏. In contrast, most transliterations of proto-cuneiform texts
involving ducks transliterate them as a sequence, MUŠEN ŠEa. Even
ignoring this inconsistency of encoding between fourth and third
millennium texts in a putative unified encoding, a further problem
would arise from the occasional transliteration that lets itself be
influence by a Sumerian reading: a few transliterations, primarily
those of composite lexical texts, use compound readings such as
UZa. If these were encoded as cuneiform, the composites would then use
the atomic signs, and would have an encoding inconsistent with their
witnesses.

Finally, the opaque labels used as sign names, while initially based
on an educated guess at the cuneiform reflex, need to be stable, and
are therefore retained even when they prove to be incorrect as
mappings to cuneiform signs; for instance, the proto-cuneiform sign 𒨾
ŠITAb3 turns out not to be related to 𒋖 ŠITA, but instead to 𒔌
SILAₓ(LAK636) \cite[220]{Wagensonner2016}; proto-cuneiform 𒡰 KAB is
not the ancestor of cuneiform 𒆏 KAB, but instead that of 𒄸 ḪUB₂ and
𒌇TUKU \cite[274]{Wagensonner2016}, a misnomer perhaps attributable to
the second millennium 𒆏–𒄸 merger. The actual relation between
proto-cuneiform and cuneiform signs is often more complex than a
one-to-one mapping; see for instance \cite[217]{Wagensonner2016} on
the development of proto-cuneiform 𒡠 IMa and 𒡡 NI₂ into cuneiform 𒋼
TE, 𒅎 IM, and 𒉎 NI₂, or \cite[220]{Wagensonner2016} on the three
proto-cuneiform ancestors of 𒔌 LAK636. In many cases, it is still
poorly understood and may remain so.

These structural incompatibilities in the analysis of character
identity, which stem from a now well-established approach to the
fourth millennium texts ultimately motivated by the avoidance of
assumptions about their language, therefore require a disunified
approach.
